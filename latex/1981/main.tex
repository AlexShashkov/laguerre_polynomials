\documentclass[a4paper,12pt]{article}
\usepackage[left=2cm,right=1cm,top=2cm,bottom=2cm]{geometry}
\usepackage{amsmath}
\usepackage{listings}
\usepackage{latexsym}
\usepackage{amsmath}
\usepackage{amssymb}
\usepackage{graphicx}
\usepackage{wrapfig}
\pagestyle{plain}
\usepackage{fancybox}
\usepackage{bm}
\usepackage[utf8]{inputenc}
\usepackage[russian]{babel}
\usepackage{floatrow}
\usepackage{pdfpages}
\usepackage{ragged2e}
\usepackage{algorithm}
\usepackage{algpseudocode}
\documentclass{article}
\usepackage[T2A]{fontenc}
\usepackage[utf8]{inputenc}
\usepackage[english, russian]{babel}
\justifying
\usepackage[inkscapeformat=png]{svg}
\usepackage[pageanchor]{hyperref}
\algblock[BLOCK]{parfor}{endparfor}

\begin{document}
\tableofcontents
\hyperpage{}

\newpage
\section{Введение}
В данной работе обобщается метод Лагерра на методы более высокого порядка. В главе 2 описывается класс методов и доказывается глобальная сходимость
для полиномов с вещественными нулями. В главе 3 обсуждается метод 4 порядка нашего класса. В 4 главе - метод порядка 3.303. 


\newpage
\section{Класс глобально сходящихся методов}

Мы ищем корни многочлена $f(x)$ степени $n$ только с действительными корнями $r_{1}, r_{2}, \cdots, r_{n}$. Для действительных чисел $x_{1}, \cdots x_{m}$, которые не являются корнями уравнения, и заданных неотрицательных целых чисел $\gamma_{1}, \cdots, \gamma_{m}$ выполняется условие:
$r=$ $\left(\gamma_{1}+1\right)+\left(\gamma_{2}+1\right)+\cdots+\left(\gamma_{m-1}+1\right)+\gamma_{m} \leqq n$. $\mathscr{C}$ - класс всех многочленов степени не выше $n$ с действительными корнями $\rho_{1}, \cdots, \rho_{l}, l \leqq n$. Зададим условие при котором многочлен $P(x)$ из класса $\mathscr{C}$, будет находиться близко к многочлену $f(x)$ и его корням:
\[
f^{(k)}\left(x_{i}\right)=P^{(k)}\left(x_{i}\right), \quad i=1,2, \cdots, m \quad \text { и } \quad \text { для каждого } i, k=0,1,2, \cdots, \gamma_{i} \eqno (2.1)
\]

Перепишем (2.1), используя обозначения Мейли. Для этого определим:
\[
S_{1}(f, x)=\frac{f^{\prime}(x)}{f(x)} \eqno (2.2)
\]
и для $k \geqq 2$
\[
S_{k}(f, x)=-\frac{d}{d x} \frac{S_{k-1}(f, x)}{k-1} \eqno (2.3
\]

Обозначим $S_{k}(f, x_{i})$ при $1 \leqq k \leqq \gamma_{i}$ как $S_{k, i}$. Определим также:
\[
F_{i}=F\left(x_{i}\right)=\frac{f\left(x_{i}\right)}{f\left(x_{m}\right)}, \quad i=1,2, \cdots, m \eqno (2.4)
\]

Пусть разность $i$-го приближения и последнего равна: $\Delta_{i}=x_{i}-x_{m}, i=1,2, \cdots, m$. А обратное соотношение корня и приближения равно $\lambda_{j}=1 /\left(x_{m}-\rho_{j}\right), j=1,2, \cdots, n$, где $\rho_{j}, j=1,2, \cdots, n$, - неопределенные корни многочлена из $\mathscr{C}$.

\\
Также можно переписать уравнение (2.1) в виде:
\[
\prod_{j=1}^{n}\left(\Delta_{i} \lambda_{j}+1\right)=F_{i}, \quad i=1,2, \cdots, m-1 \eqno (2.5)
\]

\[
\sum_{j=1}^{n}\left[\frac{\lambda_{j}}{\left(1+\Delta_{i} \lambda_{j}\right)}\right]^{k}=S_{k, i}, \quad i=1,2, \cdots, m, \quad k=1,2, \cdots, \gamma_{i} \eqno (2.6)
\]

Пусть $x_{1}, x_{2}, \cdots, x_{m}$ - это начальный набор приближений к корню $f(x)$, и $\gamma_{1}, \gamma_{2}, \cdots, \gamma_{m}$ - фиксированы, тогда процедура итераций следующая:

\begin{itemize}
    \item (1) Если $f^{\prime}\left(x_{m}\right) / f\left(x_{m}\right)$ неотрицательно, выберите решение (2.5)-(2.6) так, чтобы $\lambda_{1}$ был максимальным.

\item (2) Если $f^{\prime}\left(x_{m}\right) / f\left(x_{m}\right)$ отрицательно, выберите решение (2.5)-(2.6) так, чтобы $\lambda_{1}$ был минимальным.

\item (3) Для $\lambda_{1}^{*}$ из (1) или (2) используйте $\lambda_{1}^{*}=1 /\left(x_{m}-\rho_{1}\right)$ для нахождения $x_{m+1}=\rho_{1}=x_{m}-1 / \lambda_{1}^{*}$. $x_{m+1}$ - это наше следующее приближение к корню $f(x)$.

\item (4) Повторяйте для $j=1,2, \cdots$, используя $x_{m+j}, \cdots, x_{j+1}$, а не $x_{m}, \cdots, x_{1}$, для вычисления $x_{m+j+1}$.
\end{itemize}

Решение (2.5)-(2.6) с минимальным или максимальным $\lambda_1$, если учесть, что $p_j$ - это число повторений конкретного значения $\lambda_j$, то (2.5)-(2.6) переписываются в виде:


\[ \prod_{j=1}^{r}\left(\Delta_{i} \lambda_{j}+1\right)^{p_{j}}=F_{1}, \quad i=1,2, \cdots, m-1, \eqno (2.7)\]
\[\sum_{j=1}^{r} p_{j}\left[\frac{\lambda_{j}}{\left(1+\Delta_{i} \lambda_{j}\right)}\right]^{k}=S_{k, 1}, \quad i=1,2, \cdots, m, \quad k=1,2, \cdots, \gamma_{i} \eqno (2.8) \]



где $\sum_{j=1}^{r} p_{j}=n$. Предполагая, что $p_1 = 1$:
\[
1+\sum_{j=2}^{r} p_{j}=n \eqno (2.9)
\]
Если мы подбираем $f(x)$ и его первые две производные при некотором $x_{1}$, то есть если $m=1$ и $\gamma_{1}=2$, то наш метод - это метод Лагерра.

Обсудим порядок сходимости. Если $\gamma_{1}, \gamma_{2}, \cdots, \gamma_{m}$ фиксированы при $\gamma_{m} \geqq 2$ и $\gamma_{m} \geqq \gamma_{m-1} \geqq \cdots \geqq \gamma_{1} \geqq$ 0 и если $r_{1}$ - простой нуль многочлена $f(x)$ с единственными вещественными нулями, то порядок сходимости метода из нашего класса не меньше $p$, где $p$ - положительный вещественный корень из
\[
z^{m}-\left(\gamma_{m}+1\right) z^{m-1}-\left(\gamma_{m-1}+1\right) z^{m-2}-\cdots-\left(\gamma_{1}+1\right)=0 \eqno (2.10)
\]
\newpage
\section{Метод четвертого порядка}

Рассмотрим метод в нашем классе с $m=1$ и $\gamma_{1}=3$, то есть метод, соответствующий функции $f(x)$ и её первым трём производным в точке $x_{1}$.
Сначала отметим, что, учитывая $\gamma_{m}=\gamma_{1}=3$ и $r=3$, (2.7)-(2.9) принимают вид:


\[1+p_{2}+p_{3} & =n, \]
\[\lambda_{1}+p_{2} \lambda_{2}+p_{3} \lambda_{3} & =S_{1,1} \equiv S_{1}, \]
\[\lambda_{1}^{2}+p_{2} \lambda_{2}^{2}+p_{3} \lambda_{3}^{2} & =S_{2,1} \equiv S_{2}, \eqno (3.1)\]
\[\lambda_{1}^{3}+p_{2} \lambda_{2}^{3}+p_{3} \lambda_{3}^{3} & =S_{3,1} \equiv S_{3} \]


Полагая $p_{2}=1$ и $p_{3}=q=n-2$, мы можем решить (3.1). Если мы определим $w_{j}=\lambda_{j}-S_{1} / n$  $j=1,2,3$, (3.1) эквивалентно

\[ w_{1}+w_{2}+q w_{3}=0, \]
\[ w_{1}^{2}+w_{2}^{2}+q w_{3}^{2}=B, \eqno (3.2)\]
\[ w_{1}^{3}+w_{2}^{3}+q w_{3}^{3}=C \]

где $C=S_{3}-3 S_{2} S_{1} / n+2 S_{1}^{3} / n^{2}$ и $B=S_{2}-S_{1}^{2} / n$. Определим $L=B /(n \cdot(n-1))$ и $M=C /(n \cdot(n-1) \cdot(n-2))$, а также $z=-(n-1) \pm w_{1} / \sqrt{L}$ и $v=1 \pm w_{3} / \sqrt{L}$, где знак $\pm$ согласуется со знаком $S_{1}$.

Выражая результаты через $v$ и $z$, мы получаем:

\[v^{3}-3 v^{2}+2\left[1-\frac{M}{( \pm L \sqrt{L})}\right]=0 \eqno (3.3)\]
\[2 z^{2}+2 n z+2(n-2) v z+(n-2)(n-1) v^{2}=0 \eqno (3.4)\]
где
\[
\lambda_{1}=\frac{S_{1}}{n} \pm(n-1) \sqrt{L} \pm z \sqrt{L} \eqno (3.5)
\]

Чтобы найти $\lambda_{1}$ для нашего алгоритма, нам нужно решить кубическое уравнение (3.3) для $v$, квадратное уравнение (3.4) для $z$ и использовать (3.5) для нахождения $\lambda_{1}$. 

Для многочлена $f(x)$ с корнями степени $n>3$, подходящий $\lambda_{1}$ соответствует выбору наименьшего неотрицательного корня (3.3) и наибольшего соответствующего $z$ в (3.4). Эти действительные корни существуют.

\newpage
\section{Метод порядка 3.303} 
В данном разделе рассматривается метод с $m=2, \gamma_{2}=2$ и $\gamma_{1}=0$. Рассмотрим общие итерации из главы 2, предполагая, что $f(x)$ имеет только вещественные корни и $f\left(x_{i}\right) \neq 0, i=1,2, \cdots, m$, рассмотрим следующие предположения:

\begin{enumerate}
    \item (4.1) $x_{1}, x_{2}, \cdots, x_{m}$ таковы, что интервал, образованный $x_{1}, x_{2}, \cdots, x_{m}$, не содержит корней $f(x)$
    \item (4.2) Если $S_{m, 1}>0$, то $x_{1}, x_{2}, \cdots, x_{m}$ упорядочены по убыванию, и если $S_{m, 1}<0$, то $x_{1}, x_{2}, \cdots, x_{m}$ упорядочены по возрастанию.
\end{enumerate}
Тогда на основе этого получаем:

\[
1+\Delta_{i} \lambda_{j}>0, \quad j=1,2, \cdots, n, \quad i=1,2, \cdots, m \eqno (4.3)
\]

Если $m=2, \gamma_{2}=2, \gamma_{1}=0$, допустим, что $f(x)$ не имеет корней между $x_{1}$ и $x_{2}$ как в (4.1). Тогда (2.6)-(2.8) и (4.3) принимают вид


\[ 1+p_{2}+p_{3}=n, \eqno (4.4) \]
\[\lambda_{1}+p_{2} \lambda_{2}+p_{3} \lambda_{3}=S_{1,2} \equiv S_{1}, \eqno (4.5)\]
\[\lambda_{1}^{2}+p_{2} \lambda_{2}^{2}+p_{3} \lambda_{3}^{2}=S_{2,2} \equiv S_{2}, \eqno (4.6)\]
\[\left(1+\Delta_{1} \lambda_{1}\right)\left(1+\Delta_{1}  \lambda_{2}\right)^{p_{2}}\left(1+\Delta_{1} \lambda_{3}\right)^{p_{3}}=F_{1}, \eqno (4.7)\]
\[1+\Delta_{1} \lambda_{j}>0, \quad j=1,2,3 \eqno (4.8)\]

При $m=2, \gamma_{2}=2, \gamma_{1}=0$ и при условии (4.1) решение задачи (4.4)-(4.7) удовлетворяет с $p_{2}=1, p_{3}=n-2=q$. Получаем:

\[
w_{1}+w_{2}+q w_{3}=0 \]
\[ w_{1}^{2}+w_{2}^{2}+q w_{3}^{2}=B \]
\[\left(\Delta_{1} w_{1}+u\right)\left(\Delta_{1} w_{2}+u\right)\left(\Delta_{1} w_{3}+u\right)^{q}=F_{1} \eqno (4.9)
\]

где $B=S_{2}-S_{1}^{2} / n$, и $u=1+\Delta_{1} S_{1} / n$. Удобно ввести обозначения $L=B /(n \cdot(n-1))$, $L_{1}=\Delta_{1}^{2} L, \quad D=(n-1)+u / \sqrt{L_{1}} \quad$ и $\quad E=-1+u / \sqrt{L_{1}}, \quad$ и положить $z=$ $-(n-1) \pm w_{1} / \sqrt{L}$, а также $v=1 \pm w_{3} / \sqrt{L}$, где $\pm$ согласуется со знаком $S_{1}$ и $\Delta_{1}$. Решая (4.9) для $w_{3}$ и затем $w_{1}$, и выражая наши результаты через $v$ и $z$, мы получаем:

\[
\left(\sqrt{L_{1}}\right)^{n}\left[(n-2)(n-1) v^{2}-2(n-2) D v+2 D E\right](v+E)^{n-2}-2 F_{1}=0 \eqno (4.10) \]
\[2 z^{2}+2 n z+2(n-2) v z+(n-2)(n-1) v^{2}=0 \eqno (4.11)\]


где

$$
\lambda_{1}=\frac{S_{1,2}}{n} \pm(n-1) \sqrt{L} \pm z \sqrt{L} \eqno (4.12)
$$

Чтобы определить $\lambda_{1}$ для нашего алгоритма, мы должны решить уравнение $n$-ой степени (4.10), квадратное уравнение (4.11) для $z$, и использовать (4.12), чтобы найти $\lambda_{1}$.

Для полинома $f(x)$ степени $n>3$, имеющего только вещественные корни, если выполнены условия (4.1) и (4.2), то $\lambda_{1}$, подходящее для нашего алгоритма, соответствует выбору наименьшего неотрицательного корня (4.10) и наибольшего корня (4.11).

Мы отмечаем, не предоставляя деталей, что если $n=3$, то любой выбор корней в (4.10)-(4.11) приведет к $\lambda_{1}$, такому что на шаге (3) нашего алгоритма точно выбирается корень $f(x)$.

Теперь (4.10) - это уравнение $n$-го порядка относительно $v$, и сложно или невозможно найти явную формулу для его решения. Однако уравнение, эквивалентное (4.10), можно эффективно решить методом итерации Коши. Для описания этого мы переписываем $(4.10)$ как
\[
s(v)= & (n-2)(n-1) v^{2}-2(n-1) D v \\
& -2(v+E)\left[D-\frac{F_{1}}{\sqrt{L_{1}}\left(\sqrt{L_{1}} v+\sqrt{L_{1}} E\right)^{n-1}}\right]=0 \eqno (4.13)
\]

Мы можем найти первый неотрицательный корень $\bar{v}$ уравнения (4.10), найдя первый неотрицательный корень уравнения (4.14) методом итерации Коши. Если у $f(x)$ есть только вещественные корни, то можно показать, что $s^{\prime}(v) s^{\prime \prime}(v)<0$ для $0<v \leqq \bar{v}$. Таким образом, начиная с $v=0$, итерация Коши гарантированно сходится к $\bar{v}$.


\newpage
\section{Заключение}

В данной работе был обобщен метод Лагерра на методы более высокого порядка, которые обладают теми же желательными свойствами глобальной сходимости, что и метод Лагерра для полиномов с вещественными нулями.
Был описан подход к выводу метода Лагерра, который используется для получения многих обобщений метода Лагерра.


\end{document}